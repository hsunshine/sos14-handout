% Copyright (C) 2014 by Massimo Lauria
% 
% Created   : "2014-01-07, Tuesday 17:01 (CET) Massimo Lauria"
% Time-stamp: ""
% Encoding  : UTF-8
% ---------------------------- USER DATA ------------------------------
\def\DataTitle{Syllabus}
\def\DataTitleShort{\DataTitle}
\def\DataDate{}
\def\DataDocname{\DataTitle}
\def\DataLecturer{}
\def\DataScribe{}
\def\DataKeywords{}
\def\DataAbstract{}

% ---------------------------- PREAMBLE -------------------------------
\documentclass[a4paper,justified]{tufte-handout}
\usepackage{soscourse} % this is a non standard package


\begin{document} 
% --------------------------- DOCUMENT --------------------------------

\stressterm{Main Lecturer}: Massimo Lauria \texttt{lauria@csc.kth.se}

\stressterm{Homepage}: \url{http://www.csc.kth.se/~lauria/sos14/}

\stressterm{Lecture notes}: \url{https://github.com/MassimoLauria/sos14-handout}

\bigskip

\stressterm{Default schedule and location}: the course takes place on
Monday and Tuesday between 10:00 and 12:00 in room 4523
(Lindstedtsvägen 3). There will be temporary changes in schedule, as
in the first week. \stressterm{Please check the webpage for updates}.

\bigskip

\section{Evaluation}

The course is worth \textbf{6 credits},  and there will be \textbf{2 problem sets}
  and \textbf{scribe notes duty}.  I expect students to submit their homework
  in English,  and to produce them  with tools that have  good support
  for math  (of course  $\mathrm{\LaTeX}$ is strongly  suggested).  In
  order to succeed the students should pass \textbf{all the problem sets} and
  do a certain  amount of scribe notes (the actual  amount will depend
  on the number of people taking the course for credit).

  The idea is  that every student will \textbf{write} two  lecture notes, and
  will \textbf{review}  two lecture  notes. Each such effort  will have
  \textbf{its own deadline}.   I already scribed the first  two lecture notes
  and provided a template.

  There          is           \texttt{github}          repository          at
  \href{https://github.com/MassimoLauria/sos14-handout}{https://github.com/MassimoLauria/sos14-handout}  where  you can  post
  your scribe notes and your fix. The scheme is the following:

\begin{enumerate}
\item every writer and reviewer will fork \textbf{my} repository;
\item writers will send pull requests to me to add data to the
     repository, as often as they feel reasonable;
\item when times come reviewers will send pull requests to me.
\end{enumerate}

  It is always a good idea to  do small commits and to commit often to
  your own repository, but please do  not flood me with pull requests.
  It is fine  to send me few revisions instead  of a single monolithic
  pull request, but  please be moderate.  Your  own repository instead
  is your realm and you do as you wish (this is one of the beauties of
  distributed version control systems).

  If you  need help  with \texttt{git} and  \texttt{github} I am  sure some  of your
  colleagues will  be generous. Please  sort out \texttt{git} and  \texttt{github} as
  soon as possible.

  Lecture notes that are under  scribing/review process by one of your
  colleagues should  not be  edited by  anyone else  until “released”.
  Nevertheless \textbf{everyone} is allowed to  send pull requests with fixes
  and improvements to the released notes  and to the notation file.  A
  significant effort in  this direction may have  a \textbf{limited} positive
  influence to  the grade.  You  are \textbf{encouraged} to fix,  spell check
  and improve the  notes for the first two lectures  at any time. This
  would be particularly appreciated.


\end{document} 


%%% Local Variables:
%%% mode: latex
%%% TeX-master: t
%%% End:
