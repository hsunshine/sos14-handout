% Copyright (C) 2014 by Massimo Lauria
% 
% Created   : "2014-01-07, Tuesday 17:01 (CET) Massimo Lauria"
% Time-stamp: ""
% Encoding  : UTF-8
% ---------------------------- USER DATA ------------------------------
\def\DataTitle{Notation Syllabus}
\def\DataTitleShort{\DataTitle}
\def\DataDate{}
\def\DataDocname{\DataTitle}
\def\DataLecturer{}
\def\DataScribe{}
\def\DataKeywords{}
\def\DataAbstract{}

% ---------------------------- PREAMBLE -------------------------------
\documentclass[a4paper,justified]{tufte-handout}
\usepackage{soscourse} % this is a non standard package


\begin{document} 
% --------------------------- DOCUMENT --------------------------------


\section{Vector and Matrices}

A sequence of variables $ x_{i} $ for $i$ going from $ 1$ to $n$ can
be interpreted as a column vector $ x $ or a row vector $ x^{T} $ in
the corresponding standard basis. The inner product $ \sum_{i} x_{i}
y_{i} $ is denoted as $ x^{T}y $.  The length of a vector is
$\sqrt{x^{T}x} $ and is denoted as $ |x| $.


A double indexed set of variables $
a_{ij} $ is intepreted as a matrix $ A $. We usually denote vectors in
small caps and matrices in big caps, but this convention may change to
adapt to the context.

\begin{figure*}\label{fig:arrayconvention}
\begin{minipage}[c]{0.20\textwidth}
\[
x=\begin{bmatrix}
  x_{1}\\
  x_{2}\\
  \vdots\\
  x_{n}
\end{bmatrix}
\]
\end{minipage}
\begin{minipage}[c]{0.40\textwidth}
\[
x^{T}=\begin{bmatrix}
  x_{1} & x_{2} & \cdots & x_{n} 
\end{bmatrix}
\]
\end{minipage}
\begin{minipage}[c]{0.40\textwidth}
\[
A=\begin{bmatrix}
  a_{1,1} & a_{1,2} & \cdots & a_{1,n} \\
  a_{2,1} & a_{2,2} & \cdots & a_{2,n} \\
  & & \vdots & \\
  a_{m,1} & a_{m,2} & \cdots & a_{m,n}
\end{bmatrix}
\]
\end{minipage}
\end{figure*}

The product between a column vector $ x $ with $ m $ values and a row
vector $ y^{T} $ with $ n $, is denoted by $ xy^{T} $ and is a matrix
of rank at most 1 with $ m $ rows and $ n $ columns. The product
between two matrices $ A $ and $ B $ with $ m $ rows and $ n $ columns
is denoted as $\pointwiseprod{A}{B}$.

\begin{figure*}
\begin{minipage}[b]{0.45\textwidth}
\begin{equation*}
  xy^{T}=\begin{bmatrix}
  x_{1}y_{1} & x_{1}y_{2} & \cdots & x_{1}y_{n} \\
  x_{2}y_{1} & x_{2}y_{2} & \cdots & x_{2}y_{n} \\
  & & \vdots & \\
  x_{m}y_{1} & x_{m}y_{2} & \cdots & x_{m}y_{n}
\end{bmatrix}
\end{equation*}
\end{minipage}
\begin{minipage}[b]{0.45\textwidth}
  \begin{equation*}
    \pointwiseprod{A}{B}=\begin{bmatrix}
      a_{1,1}b_{1,1} & a_{1,2}b_{1,2} & \cdots & a_{1,n}b_{1,n} \\
      a_{2,1}b_{2,1} & a_{2,2}b_{2,2} & \cdots & a_{2,n}b_{2,n} \\
      & & \vdots & \\
      a_{m,1}b_{m,1} & a_{m,2}b_{m,2} & \cdots & a_{m,n}b_{m,n}
    \end{bmatrix}
  \end{equation*} 
\end{minipage}
\end{figure*}

The notations $ x \geq y, x \leq y, x < y, x > y$ for vectors mean
that the corresponding inequalities hold pointwise (\ie, if $ x \leq y
$ then $ x_{i} \leq y_{i} $ for every $ i $). We assign a similar
meaning to notations $ A \geq B, A \leq B, A < B, A > B$.
%
The notation $ A \succeq 0 $ means that $ A $ is
\introduceterm{positive semidefinite}, the notation $ A \succ 0 $ that
$ A $ is \introduceterm{positive definite}.

\section{Linear programs}

We denote linear programs in one of this fashion.

\begin{figure}
\begin{minipage}[t]{0.5\textwidth}
\begin{alignat*}{2}
  \maximize c^{T}x\\
  \subjectto Ax \leq b
\end{alignat*}
\end{minipage}
\begin{minipage}[t]{0.5\textwidth}
\begin{alignat*}{2}
  \maximize c^{T}x \\
  \subjectto Ax = b\\
  & x \geq 0
\end{alignat*}
\end{minipage}
\end{figure}

For a linear program such that we denote as $ \sol{P} $ the set of
feasible solution for that program, and as $ \sol{P}_{I} $ the convex
hull of set of integer feasible solutions of the same program, \ie, $
\sol{P}_{I} = \mathsf{convexhull}(\ZZ^{n} \cap \sol{P}) $.

\section{Semidefinite programs}

We denote semidefinite programs (\sdp) in one of this way.

\begin{figure}
\begin{minipage}[t]{0.5\textwidth}
\begin{alignat*}{2}
  \maximize c^{T}x\\
  \subjectto \pointwiseprod{A_{1}}{X} \leq b_{1}\\
  & \pointwiseprod{A_{2}}{X} \leq b_{2}\\
  & \vdots \\
  & \pointwiseprod{A_{\ell}}{X} \leq b_{\ell}\\
  & X \succeq 0
\end{alignat*}
\end{minipage}
\begin{minipage}[t]{0.5\textwidth}
\begin{alignat*}{2}
  \maximize \sum_{i}c_{i} (v_{0}^{T}v_{i}) \\
  \subjectto \sum_{i,j} a_{i,j} (v^{T}_{i}v_{j})= b_{1}\\
  & \sum_{i,j} a^{2}_{i,j} (v^{T}_{i}v_{j})= b_{2}\\
  & \vdots \\
  & \sum_{i,j} a^{\ell}_{i,j} (v^{T}_{i}v_{j})= b_{\ell}\\
  & |v_{0}| = 1
\end{alignat*}
\end{minipage}
\end{figure}

There is no much difference if the constraints are given in form of
equations of in form of inequalities.  For semidefinite programs such
that we denote as $ \sol{P} $ the set of feasible solution for that
program, and as $ \sol{P}_{I} $ the convex hull of set of integer
feasible solutions of the same program, \ie, $ \sol{P}_{I} =
\mathsf{convexhull}(\ZZ^{n} \cap \sol{P}) $.


\section{Other notation}

For $ n\leq 0 $ and $ k\leq n $,
\begin{equation*}
  \binom{n}{\leq k} = \sum^{k}_{i=0}\binom{n}{i} 
\end{equation*}

For a set $ S $ and integer $k$,
\begin{equation*}
  \binom{S}{k}= \{ T \subseteq S, |T|=k  \} \qquad \binom{S}{\leq k} =
  \bigcup^{k}_{i=0} \binom{S}{i}.
\end{equation*}


% ------------------------- EPILOGUE ------------------------------
% \bibliography{soscourse}
% \bibliographystyle{alpha}

\end{document} 


%%% Local Variables:
%%% mode: latex
%%% TeX-master: t
%%% End:
