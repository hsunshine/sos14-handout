% Copyright (C) 2014 by Massimo Lauria
% 
% Created   : "2014-01-07, Tuesday 17:01 (CET) Massimo Lauria"
% Time-stamp: "2014-01-07, 17:53 (CET) Massimo Lauria"
% Encoding  : UTF-8

% ---------------------------- PREAMBLE -------------------------------
\documentclass[a4paper,symmetric,justified,10pt]{tufte-handout}

% ---------------------------- PACKAGES -------------------------------
% GENERAL
\usepackage[utf8]{inputenc}
\usepackage[english]{babel}
\usepackage{tgtermes}
\usepackage{parskip}
\usepackage{xspace}
\usepackage{pdfsync}
\usepackage{lipsum}   % filler text

% MATH
\usepackage{soscourse} % this is a non standard package
% ALGORITHM
% \usepackage{algorithm}
% \usepackage[noend]{algpseudocode}
% \usepackage{listings}
% GRAPHICS
% \usepackage{graphicx,graphics,epsfig,subfigure}
% \usepackage{pgf}
% \usepackage{tikz}
% \usetikzlibrary{arrows,shapes,backgrounds,snakes}
% ---------------------------------------------------------------------

% ---------------------------- USER DATA ------------------------------

\title{Lecture note 1}
\author{Lecturer: Massimo Lauria\hspace{3em}Scribe: Massimo Lauria}
\date{\today}

\hypersetup{%
  pdfauthor={Massimo Lauria},
  pdftitle ={Lecture note 1},
  bookmarks=true,
  bookmarksnumbered=true,
  pagebackref=true,
  colorlinks=true,
  linkcolor=blue,
  citecolor=red,
  hyperfootnotes=true,
  pdftex,
  pdfsubject={integer programming relaxation, sum of squares},
  pdfkeywords={}
}

% ---------------------------------------------------------------------



% ---------------------------- MACROS ---------------------------------
% Define your macro here.
% ---------------------------------------------------------------------



\begin{document} % ----------- BEGIN DOCUMENT -------------------------

% ---------------------------- TITLE/ABSTRACT PAGE --------------------


\maketitle
%
\sidenote{aaaaaaaaaa}

\begin{abstract}
  \textbf{Summary:} this is a three four lines (top!) description of
  the content of the lecture.
\end{abstract}


\lipsum[1]


% \begin{fullwidth}
% \lipsum[2]
% \end{fullwidth}

\lipsum[3]\footnote{\lipsum[4]}

\newpage

\lipsum[4]

\newpage

\lipsum[5]


% Uncomment this if you want the paper to start in the next page.
% \thispagestyle{empty}
% \clearpage
% ---------------------------------------------------------------------


% --------------------------- CONTENT --------------------------------

\bibliography{theoryofcomputing}
\bibliographystyle{alpha}

\end{document} % ------------- END DOCUMENT --------------------------


%%% Local Variables:
%%% mode: latex
%%% TeX-master: t
%%% End:
